%PREÁMBULO
\documentclass[a4paper]{article}
\usepackage[spanish]{babel}
\usepackage[utf8]{inputenc}
\usepackage[pdftex]{graphicx}
\usepackage{hyperref} %Permite poner links
\parindent=0.5cm %Modificar el tamaño de la sangría
\parskip=0.5cm %Modificar el espaciado entre párrafos
\usepackage{float} %Movimiento de tablas y gráficos
\usepackage[lmargin=2.5cm,rmargin=2.5cm,top=2.5cm,bottom=2.5cm]{geometry} %Márgenes
\usepackage{natbib} %Paquete para que en el apartado de referencias no aparezcan los nombres por duplicado.

%1.Encabezado
\usepackage{fancyhdr}
\pagestyle{fancy}
\fancyhead{} %Elimina el texto predeterminado del encabezado fancy
\fancyhead[C]{El camarón \textit{Palaemon serratus}} %Texto del encabezado
\renewcommand{\headrulewidth}{0.25pt} %Grosor de la línea

%2.Pie de página
\fancyfoot{} %Elimina el texto predeterminado del pie de página
\fancyfoot[R]{\thepage} %Comando para la numeración de página
\fancyfoot[L]{Jorge Garrido Bautista}

%COMIENZO DEL DOCUMENTO
\begin{document}

%Página inicial para el título del artículo y el autor
\begin{titlepage}
\begin{center}
\vspace*{5\baselineskip} %Espacio vertical
{\LARGE \textbf{Biología del camarón \textit{Palaemon serratus}\\(Pennant, 1777)}}
\vspace*{5\baselineskip}
\vfill %Este comando rellena con huecos vacíos el espacio entre la imagen y el último texto
{\large JORGE GARRIDO BAUTISTA}\\
\vspace*{4\baselineskip}
Curso de LaTeX y Git. Darwin Eventur\\
\textit{Texto adaptado a partir de mi TFM}
\end{center}
\end{titlepage}

%Segunda página inicial para el abstracts
\begin{titlepage}
\begin{abstract}
Repositorio:\\
\url{https://github.com/JorgeGarridoBautista/Proyecto_Final}\\
\\El camarón \textit{Palaemon serratus} (Pennant, 1777) es un crustáceo decápodo de la familia Palaemonidae que se distribuye por las costas del este del océano Atlántico y mar Mediterráneo. Su rápido desarrollo y su larga secuencia de estadios larvarios son características de interés para el estudio de los genes que intervienen, de forma directa o indirecta, en los procesos de muda o ecdisis y en la determinación o diferenciación sexual. En el presente artículo se hablará brevemente de todos estos aspectos.\\
\\Palabras clave: muda, ecdisis, biología básica, camarón, \textit{Palaemon serratus}
\end{abstract}
\end{titlepage}

%Comienzo del CUERPO DEL ARTÍCULO
\begin{titlepage}
\tableofcontents
\end{titlepage}

\section{Biología básica de \textit{Palaemon serratus}}
\subsection{Morfología y reproducción}
El camarón \textit{Palaemon serratus} (Pennant, 1777) es un crustáceo decápodo de la familia Palaemonidae que se caracteriza por su color casi transparente interrumpido por un pereion de líneas longitudinales rojizas que empiezan en el rostro y terminan en el borde posterior. Además de esto, también se distingue de otras especies coetáneas por la presencia de anillas amarillas y rojas en las patas \citep{Zariquiey1968}.\\
Blaf fdfs
 
\subsection{Desarrollo larvario}

\section{La muda o ecdisis de \textit{Palaemon serratus}}

\section{Bibliografía}
\bibliography{Biblio}
\bibliographystyle{apalike} %Bibliografía estilo APA. Cuando se introducen las referencias bibliográficas (.bib) hay que actualizar varias veces para que salga el formato correcto y todas las citas. Se hace presionando F6 y F11 varias veces seguidas.

\end{document}
